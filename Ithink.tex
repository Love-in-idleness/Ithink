\documentclass[fontset=windows, 12pt, a4paper]{article}
\usepackage{amsmath, amsthm, amssymb, graphicx}
\usepackage{mathrsfs,ctex, indentfirst}
\usepackage[bookmarks=true, colorlinks, citecolor=blue, linkcolor=black]{hyperref}
\usepackage[left=1in, right=1in, top=1in, bottom=1in]{geometry}
\usepackage{listings}

\setlength{\parindent}{0em}
\newtheorem{theorem}{定理}[section]
\date{}
\title{Ithink}
\begin{document}
\maketitle
%\tableofcontents

\section*{一、假定}
\subsection*{(1)自然界可独立于人而存在}

我们知道,人类个体思维主要来自人类个体大脑。
在自然界中,
人类个体思维实质上对应人类个体的大脑生理结构,
及相关电信号、化学作用等自然界中的变化。
对其他非人类个体,答案也是一样的。
由此得知个体思维实质是自然界的,即:
$$\pmb{\mbox{个体思维} \subset \mbox{自然界}}$$

个体思维所能接受的自然界作用是有限的,
个体思维认知自然界的过程需要靠外部设备转化,
这个转化是多步的,比如:\par
\begin{center}
    ……
    $\Rightarrow$\par
    叶子反射无法吸收的可见光
    $\Rightarrow$\par
    眼睛接受可见光并加工成信号
    $\Rightarrow$\par
    神经系统接收并计算信号
    $\Rightarrow$\par
    认知到这是叶子+其他
\end{center}
特别的,还有条需要单独说明的路径:\par
\begin{center}
    ……
    $\Rightarrow$\par
    他人的思维
    $\Rightarrow$\par
    ……
    $\Rightarrow$\par
    言语、文字、音律等
    $\Rightarrow$\par
    ……
    $\Rightarrow$\par
    神经系统接收并计算信号
    $\Rightarrow$\par
    个体思维
\end{center}

我们将外部设备这个概念抽象化,
将其化为一种运算:\par
\centerline{\textbf{
    外设 $_1($ 外设 $_2($ 外设 $_3(\cdots$
    外设 $_n($ 可观测自然界 $)\cdots)))=$
    可接收信号
}}
把外设的组合视为外设路径,即:\par
\centerline{\textbf{
    外设路径(可观测自然界)=可接收信号
}}
无论在个体还是集体中,路径都是不一的,
不一样的路径得到的结果也是不一样的。
对个体的单次接受,一般有:\par
\centerline{\textbf{
    某条外设路径(个体可观测自然界的某些元素)=某次信号
}}
在此之后,个体思维处理了这次的信号,即:\par
\centerline{\textbf{
    个体思维(某次信号)=某次认知
}}
我们从感性上做一个结论:
个体思维的表现为认知。
另外,如前面所说,思维的实质是自然界的
从而:\par
\centerline{\textbf{
    某次信号(个体思维)=个体思维*
}}
个体思维在物理上发生变化。

内含“相信”与“逻辑”两重主概念。

人类通过有限的感知获取世界的信息,
我们将过去、现在、将来所有可及可能获取的信息取


人类思维

% \bibliography{info}
\bibliographystyle{ieeetr}

\end{document}
