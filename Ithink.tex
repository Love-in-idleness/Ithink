\documentclass[fontset=windows, 12pt, a4paper]{article}
\usepackage{amsmath, amsthm, amssymb, graphicx}
\usepackage{mathrsfs,ctex, indentfirst}
\usepackage[bookmarks=true, colorlinks, citecolor=blue, linkcolor=black]{hyperref}
\usepackage[left=1in, right=1in, top=1in, bottom=1in]{geometry}
\usepackage{listings}

\setlength{\parindent}{0em}
\newtheorem{theorem}{定理}[section]
\date{}
\title{Ithink}
\begin{document}
\maketitle
%\tableofcontents

\section*{前言}
这什么都不是,这只是狂言妄语;这里空空如也,能看到的只有无知。


\section{心智}

在开始思考别的事物前,我们首先要来思考自我的心智。这个过程是痛苦且矛盾的,我们在一开始什么都无法确定,因为我们不知道自然界存不存在,我们甚至不清楚自身是什么;这个过程是必要且需最先进行的,因为把它放到后面难免出现“循环论证”这类让人难以信服的步骤。我们在开始一切的时候,甚至不需要前提,或者说,我们要一些不得不承认的前提,而这些前提(只要是我能想到的),我会在这里一一列出。

另外我这里会先讨论“我”的问题,非“我”乃至非人的问题我将在后面说明。

我先要说明:
\textbf{“我”是具有心智的,即“我”能进行心理活动。}

这句话就算不是绝对的,但至少适用于绝大部分的人,它就是这样存在着(无论是否是被操控的)。心理活动活动需要它所存在的空间:\textbf{意识、无意识等};心理活动则包括\textbf{情绪、信任、思维、认知、想象等}

\subsection{心智空间}
我们将看到,心智空间的分类是基于自我察觉的。然而自我察觉是不是非有则无?不见得,以笔者为例,有时候身体不舒服,但把意识集中干事时,不适感慢慢消失了。可见有意识的痛苦转移到无意识上了。现在我们还很难做到自主的把心理活动在无意识和有意识间转换,但以后呢?

于是,我们在这里提出的意识与无意识都只是沿用精神分析留给我们的名词,来方便我们叙述。在后续操作中,我们一般不对其进行严格划分,而一般用心智空间这个名词。无意识和意识之间的壁垒不是绝对的。

前意识。

\subsubsection{意识}
拥有意识,即“我”认识到自我,能进行自我觉察。

精神分析学对此有详细的解释,但是在这里,我们是不需要这些的。因为“我”现在是否拥有自我,“我”自然知道。

\subsubsection{无意识}
无意识是自己无法察觉其中发生了怎么样的心理活动的心智空间。

和意识一样,我们只是在这里写出它,不加过多解释。

\subsubsection{其他}
会有其他空间存在吗?即是否存在可自由选择察觉或不察觉的空间存在?我不敢下判断。即使现在没有,未来也可能有。

\subsection{活动}

\subsubsection{情绪}
情绪,是心智的波动,简单来说,就是喜怒哀乐。

我们且不考虑这些情绪从何处来,因为这些解释是后面才能进行的,且信教者与无神论者的结果是不一致的。所以,我们只是承认这些事物的存在。

情绪有积极的,也有消极的。它们相对,但并不是非黑即白的,即非完全对立的。只是这两重概念能囊括最多的事物,于是我们划分出积极与消极。

对于积极的事物,人们总是乐意见到的;对于消极的,人们会选择逃避。积极与消极是有质和量的,人们因为它们间的质与量最后认定一个事物本身的好恶。

\subsubsection{信任}
信任:建立在对他人的意向或行为的积极预期基础上而敢于托付(愿意承受风险) 的一种心理状态。\cite{ref1}

我们可以知道,信任是由好的情绪发展来的。这里,我要指出,我们最大的信任给了重复,我们认为重复的事物仍将重复。我们自己往往不会意识到这一点,这些心智活动在无意识中也可以发生。
信任是“我”的思维的前置概念,其优先级高于逻辑。



\subsubsection{思维}
千百万年,逻辑在一次次重复下,人们信任了它,人们认为逻辑具有可靠性,且在公理正确选取下具有一致性。
无意识的思维浮现为直觉。

思维是被语言和逻辑符号限制的。

\subsection{认知}


\subsection{想象}
“我”想象得出的东西无法超出“我”的认知

\section{假定}
\subsection{自然界可独立于人而存在}

我们知道,人类个体思维主要来自人类个体大脑。在自然界中,人类个体思维实质上对应人类个体的大脑生理结构,及相关电信号、化学作用等自然界中的变化。对其他非人类个体,答案也是一样的。由此得知个体思维实质是自然界的,即:
$$\pmb{\mbox{个体思维} \subset \mbox{自然界}}$$

个体思维所能接受的自然界作用是有限的,个体思维认知自然界的过程需要靠外部设备转化,
这个转化是多步的,比如:\par
\begin{center}
    ……
    $\Rightarrow$\par
    叶子反射无法吸收的可见光
    $\Rightarrow$\par
    眼睛接受可见光并加工成信号
    $\Rightarrow$\par
    神经系统接收并计算信号
    $\Rightarrow$\par
    认知到这是叶子+其他
\end{center}
特别的,还有条需要单独说明的路径:\par
\begin{center}
    ……
    $\Rightarrow$\par
    他人的思维
    $\Rightarrow$\par
    ……
    $\Rightarrow$\par
    言语、文字、音律等
    $\Rightarrow$\par
    ……
    $\Rightarrow$\par
    神经系统接收并计算信号
    $\Rightarrow$\par
    个体思维
\end{center}

我们将外部设备这个概念抽象化,将其化为一种运算:\par
\centerline{\textbf{
    外设 $_1($ 外设 $_2($ 外设 $_3(\cdots$
    外设 $_n($ 可观测自然界 $)\cdots)))=$
    可接收信号
}}
把外设的组合视为外设路径,即:\par
\centerline{\textbf{
    外设路径(可观测自然界)=可接收信号
}}
无论在个体还是集体中,路径都是不一的,不一样的路径得到的结果也是不一样的。对个体的单次接受,一般有:\par
\centerline{\textbf{
    某条外设路径(个体可观测自然界的某些元素)=某次信号
}}
在此之后,个体思维处理了这次的信号,即:\par
\centerline{\textbf{
    个体思维(某次信号)=某次认知
}}
我们从感性上做一个结论:个体思维的表现为认知。另外,如前面所说,思维的实质是自然界的,从而:\par
\centerline{\textbf{
    某次信号(个体思维$_{m-1}$)=个体思维$_m$
}}
个体思维在物理上发生变化。

实际上 $\phi(A)=B+C$,运算无法连续进行

认知也是并行的:\par
$\sum$某次认知=某个时刻的认知

思维是可以再细分的,但我们还是将其作为复合体以方便描述。



人类思维


\begin{thebibliography}{99}  
    \bibitem{ref1}《信任的心理和神经生理机制》「张宁,张雨青,吴坎坎」“2011.10” 
    \bibitem{ref2}Arandjelović R, Zisserman A, Three things everyone should know to improve object retrieval, Computer Vision and Pattern Recognition (CVPR), 2012 IEEE Conference on, IEEE, 2012: 2911-2918.  
    \bibitem{ref3}Lowe D G. Distinctive image features from scale-invariant keypoints, International journal of computer vision, 2004, 60(2): 91-110.  
    \bibitem{ref4}Philbin J, Chum O, Isard M, et al. Lost in quantization: Improving particular object retrieval in large scale image databases, Computer Vision and Pattern Recognition, 2008. CVPR 2008, IEEE Conference on, IEEE, 2008: 1-8.  
\end{thebibliography}

% \bibliography{info}
\bibliographystyle{ieeetr}

\end{document}
